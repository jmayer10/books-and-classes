\documentclass[12pt,fullpage]{article}
\usepackage{microtype}
\usepackage{listings}
\usepackage{graphicx}
\usepackage{latexsym}
\usepackage{longtable}
\usepackage{setspace}
\begin{document}

\title{The New Trading For A Living\\Dr. Alexander Elder}
\maketitle

\newpage

\onehalfspace
\section*{Introduction}
- Psychology is the most important factor in trading. Yours and others.\\
- The three pillars of successful trading are \textbf{psychology}, \textbf{market analysis}, and \textbf{risk management}. Good record-keeping ties them together.\\
- Chart patterns reflect crowd behavior. Technical analysis is applied social psychology.\\
- Keep an eye on commissions and slippage; they sum to significant values over time and frequency.\\
\section*{Chapter 1: Individual Psychology}
- Our fantasies influence our behavior, even if we aren't consciously aware of them.\\
- Amateurs neither expect to lose, or are prepared to manage losing trades.\\
- Three essential components of trading: \textbf{sound psychology}, \textbf{a logical trading system}, and \textbf{an effective risk management plan}.\\
- Self-sabotage is a primary source of failure.\\
- At the early stages of your career you may have to devote as much time to analyzing yourself as you do to analyzing the markets.\\
- Profits give traders an emotional high and \textbf{a feeling of power}. (This is me.)\\
- Develop a method for analyzing the markets.\\
\section*{Chapter 2: Mass Psychology}
- Do not become anchored to a price. Just because a stock was at \$100 and is now at \$50 does not make it a bargain. Look at the valuation and for the path of least resistance (up or down).\\
- There is a crowd of traders (big money, little money, smart money, dumb money, institutional money, private money, long term investors, and short term traders) behind every pattern on the screen.\\
- Each brings their own set of rules and strategies; cumulatively represented in the chart pattern (the minority game).\\
- Chart patterns reflect swings of mass psychology in the financial markets.\\
- The goal is to discover the balance of power between bulls and bears and bet on the winning group.\\
- When you analyze the market you are looking at crowd behavior.\\
- Don't buck a trend because "prices are too high", know what kind of market you're in and follow the path of least resistance.\\
- Stock manipulation is just \textbf{marketing,} and the ticker price is the best advertisement. It's surprising how much stock you can sell into a down trend.\\
- The \emph{Wisdom of the Crowd} only works if individuals make independent decisions and aren't influenced by the decisions of group leaders.\\
- \textbf{Bull Market Psychology:} Bulls feel optimistic and don't mind paying high prices. Bears feel afraid and only agree to sell (short) at every increasing prices.\\
- \textbf{Bear Market Psychology:} Bears feel optimistic and don't mind selling short at lower prices. Bulls are fearful and agree to buy only at a discount.\\
- Fear is 3x stronger than greed, according to Kahneman.\\
- Winners feel rewarded when price moves in their favor, and losers feel punished when it moves against them. Crowd members remain blissfully unaware that by focusing on price they create their own leader (price). Traders who feel mesmerized by prices create their own idols.\\
- \textbf{Final Thought:} \emph{Successful trading stands on three pillars. You need to analyze the balance of power between bulls and bears. You need to practice good money management. You need personal discipline to follow your trading plan and avoid getting high or depressed in the markets.}\\
\section*{Chapter 3: Classical Chart Analysis}
- Classical charting can be very subjective. Need to take only its objective components. Horizontal \textbf{support} and \textbf{resistance} lines.\\
- \underline{Bar Chart/Candlesticks}
\begin{itemize}
   \itemsep0em
   \item \textbf{Opening Price:} Reflects amateur's opinions. Creates extremes and is typically close to the high or low of the bar.
   \item \textbf{Closing Price:} Reflects the opinion of professionals. If prices closed higher than they opened pros are bullish, if closed lower pros were bearish.
   \item \textbf{High Price:} Represents the maximum power of bulls during that bar. 
   \item \textbf{Low Price:} Represents the maximum power of bears during that bar.
   \item \textbf{H/L Difference:} Reflects intensity of battle between bulls and bears (vol). Pays to enter during short/average bars.
\end{itemize}
- \textbf{Support:} a price level where buying is strong enough to interrupt or reverse a downtrend.\\
- \textbf{Resistance:} a price level where selling is strong enough to interrupt or reverse an uptrend.\\
- Traders buy and sell at these levels making them self-fulfilling prophecies.\\
- Support and resistance exist because people have memories.\\
- The longer prices stay in one congestion zone, the stronger the emotional commitment of bulls and bears to that area (anchoring). These zones become S/R levels.\\
- Strength of these zones/levels depends on length, height, and volume.\\
- S/R are more important on long term charts than on short term ones.\\
- True breakouts don't dip back into S/R levels. If it does it's a false breakout and the trend could reverse.\\
- \textbf{On Placing Stops:} try not to place them around round numbers as that is the tendency of other traders and it causing clustering at those levels.\\
- Most breakouts are false, typically marked by light volume and a divergence between prices and indicators. Can use mean reversion strategies to profit off them.\\
- A pattern of higher tops and higher bottoms defines uptrends, while a pattern of lower bottoms and lower tops defines downtrends.\\
- Markets spend most of the time in trading ranges making true trends difficult to detect.\\
- Use tight stops in ranges and more loose stops in trends.\\
- Ranges are where you play \textbf{mean reversion}, in trends ride the \textbf{momentum}.\\
- Market analysis should occur over multiple time frames. What looks like a trend in one frame could look like a range blip if you zoom out.\\
- Single large bars, know as \textbf{finags} or \textbf{kangaroo tails} can be reliable signals. The next bar will tell whether a trend is starting or if the trend was rejected.\\
\section*{Chapter 4: Computerized Technical Analysis}
\underline{Three Major Indicator Groups:}
\begin{itemize}
   \itemsep0em
   \item \textbf{Trend-following:} Work best when markets are moving, but quality deteriorates when markets go flat. They are lagging indicators.
   \item \textbf{Oscillators:} Catch turning points in flat markets but give premature and dangerous signals when markets begin to trend. They are leading indicators.
   \item \textbf{Miscellaneous:} Provide insights into mass psychology of market. 
\end{itemize}
\underline{Moving Averages:}\\
- Simple moving averages give equal weight to all prices. Exponential averages are preferred because they weight recent prices higher.\\
- Choosing a length is difficult because there is no "best" number. Consider numbers like 22 as that is the average number of trading days in a month, anything below 8-days defeats the purpose of the MA.\\
- Crossovers of long and short MAs are a common signal to buy/sell.\\
- Moving averages often serve as \textbf{support and resistance} levels. A rising MA tends to serve as a floor below prices, and a falling MA serves as a ceiling above them.\\
- MAs can also be applied to \textbf{indicators}.\\
- Slope of MA is most important because it tells you the direction of the indicator. Consider taking the slope over several steps.\\
- Quote from Author: "I call the space between two EMAs the value zone."\\
- \textbf{Channel:} consists of two lines drawn parallel to a MA.\\
\\
\underline{MACD: Moving Average Convergence-Divergence:}\\
- Consists of three EMAs, and appears as two lines whose crossovers give trading signals.\\
- Algo:
\begin{enumerate}
   \itemsep0em
   \item Calculate 12-day EMA of closing prices.
   \item Calculate 26-day EMA of closing prices.
   \item Subtract 26-day EMA from 12-day EMA, and plot their difference as a solid line. This is the fast MACD line.
   \item Calculate a 9-day EMA of the fast MACD line from step 3, and plot the result as a dashed line. This is the slow Signal line.
\end{enumerate}
- Crossovers of fast and slow lines can be used as signals.\\
- \textbf{MACD-Historgram} is equal to MACD line minus Signal line.\\
- When the slope of MACD-Histogram moves in the same direction as prices, the trend is safe.\\
- The slope of MACD-Histogram is more important than its position above are below the centerline. Trade in the direction of the slope, buy when going up, sell when going down.\\
- Rank of MACD-Hist can show when bulls (100) and bears (0) are the strongest.\\
- \textbf{Divergences} occur when prices and technical indicators, well diverge. Ex. prices are falling but indicators say to buy.\\
- \textbf{The Hound of the Baskervilles:} The signal was given by the lack of expected action - by the lack of barking.\\
\\
\underline{Oscillators}\\
- Help catch turning points by measuring extremes in momentum and when they might break.\\
- Identify overbought and oversold conditions.\\
- Usually faded, i.e. a bullish extreme is sold short.\\
- Typically done over short time periods, 3-10 days.\\
- Work well in trading ranges, but can give premature and dangerous signals when a new trend occurs.\\
- Combine Oscillators with long-term trend signals as they will give contradictory signals during a trend/momentum.\\
\underline{Stochastic:} tracks the relationship of each closing price to the recent high-low range.\\
- Consists of two lines: a fast line called \%K and a slow line called \%D.\\
- Measures the capacity of bulls or bears to close the market near the upper or lower edge of the recent range.\\
- Designed to fluctuate between 0 and 100. Reference lines usually drawn at 20 percent and 80 percent levels to mark overbought and oversold areas.\\
- Divergence occurs when prices rally/fall but Stochastic traces a higher/lower bottom/top.\\
- Overbought and Oversold conditions occur when the reference levels are breached.\\
- When both Stochastic lines are headed in the same direction, they confirm the short-term trend.\\
\underline{Relative Strength Index: RSI}\\
- Measures strength by monitoring changes in its closing prices.\\
- Fluctuates between 0 and 100 with reference lines at 30 and 70 percent.\\
- Only a single line as opposed to two lines in Stochastic.\\
- Many of the same rules for divergences and levels as Stochastic.
\section*{Chapter 5: Volume and Time}
\underline{Psychology:}\\
- The act of buying or selling creates an emotional commitment in most people. They crave to be right\\
- The greater the volume, the more pain and euphoria in the market.\\
- Trends can exist for a long time on moderate volume but can expire after a burst of volume.\\
- Losses can be like boiling a frog, small loses over several days add up without notice.\\
- Falling volume is a sign that the trend is about to reverse. A burst of extremely high volume also gives a signal that a trend is nearing its end (Create Kangaroo tails in prices).\\
- Volume spikes are more likely to signal an imminent reversal of a downtrend than an uptrend, because \textbf{fear} is a very powerful but short-term emotion.\\
- A breakout on low volume shows little emotional commitment to a new trend.\\
- As a rule of thumb, \textbf{high volume} is a least 25\% above the average for the past two weeks, while \textbf{low volume} is at least 25\% below average.\\
- Can use EMAs to smooth at volume and observe trends.\\
- Wall Street saying: \emph{"It takes buying to put prices up, but they can fall of their own weight."}\\
\\
\underline{Volume-Based Indicators:}\\
- Most simple indicator is a 5-day EMA, in which a rising EMA affirms the current price trend while declining points to weakness.\\
- \textbf{On-Balance Volume (OBV):} A running total of volume. Volume on an up day is added, and volume on a down day is subtracted from the total.\\
- High volume represents and intensifying of psychology whether or not the price movement is large as a result.\\
- OBV gives it's strongest signals when it diverges from prices. If prices rally to a new high buy OBV rallies to a lower high, the divergence is bearish and gives a sell signal.\\
- \textbf{An Example of Creating New Indicators:}
\begin{itemize}
   \itemsep0em
   \item Using OBV you can create the \textbf{Net Field Trend Indicator} and the \textbf{Climax} indicator. For NFTI classify the OBV pattern as rising (+1), falling (-1), or neutral (0). The climax indicator is then the NFTI for all stocks in an index for example.
\end{itemize}
- \textbf{Accumulation/Distribution:} A leading cumulative indicator that tracks the relationship between opening and closing prices in addition to volume.\\
- A/D credits bulls/bears with only a fraction of each day's volume. \[A/D = \frac{Close - Open}{High - Low} * Volume\]
- For OBV and A/D the pattern of highs and lows is important, while the absolute level simply depends on the starting date.\\
- As with OBV A/D gives the best signals when it diverges from prices.\\
\textbf{Open Interest:} is the number of contracts held by buyers or owed by short sellers in any derivative market.\\
- OI rises when new positions are being created and falls when positions are being closed.\\
- If OI increases during a trend then the supply of losers is growing, and the current trend is likely to persist. An increase in OI gives a green light to an existing trend. Flattening OI is a yellow light, and declining OI is red.\\
\\
\underline{Time:}\\
\textbf{Cycles:} Long-term price cycles are a fact of economic life. Short-term cycles are most likely noise, don't use them to predict minor turning points.\\
\textbf{Indicator Seasons:} Modeling the utility of an indicator by using seasons (spring, summer, fall, winter) to describe the current state of the indicators utility.\\
- We can define the seasons of many indicators by two factors: their slope as well as their position above or below the centerline.\\
\textbf{Example with MACD Histogram:}
\begin{center}
   \begin{tabular}{|c|c|c|c|}
      \hline
      Indicator Slope & Position Relative to Centerline & Season & Preferred Action \\
      \hline
      Rising & Below & Spring & Go Long \\
      \hline
      Rising & Above & Summer & Starting Selling \\
      \hline
      Falling & Above & Fall & Go Short \\
      \hline
      Falling & Below & Winter & Start Covering \\
      \hline
   \end{tabular}
\end{center}
- Spring is the best time to buy because most traders expect the winter to return, and are afraid to buy with memories of a downtrend still fresh in their minds.\\
- This is only a model and does not follow perfectly. The unexpected like Indian Summers and early frosts still happen.\\
- \textbf{Factor of Five:} 4.5 weeks in a month, 5-days in a week, 6.5 hours in a day, 6 10min periods in an hour. These are a nice set of time frames to look at because they scale well with each other.\\
- \textbf{Swing Trading:} Expected duration of a trade is measured in days, sometimes weeks. Only disadvantage is you will miss the major trends of longer term investing.
\section*{Chapter 6: General Market Indicators}
- \textbf{New High - New Low Index:} Difference between number of stocks in an index and all times highs and those at all time lows.\\
- Can be looked at in multiple timeframes and have trends and divergences like other indicators.\\
- \textbf{Stocks above 50-Day MA:} Self explanatory as an indicator.\\
- \textbf{Advance/Decline:} tracks the degree of mass participation in rallies and declines by each day adding up the number of stocks that closed higher and subtracting the number of stocks that closed lower.\\
- Typically A/D takes a broader view and looks at all stocks listed on a exchange while NH-NL focuses on the top stocks in an index.\\
- Sentiment (what is the press saying) can also be tracked as a way to gauge whether a bull/bear market will continue.
\section*{Chapter 7: Trading Systems}
- Along with backtesting consider manual testing as well. Run the market data and force yourself to make trade decisions based on the available information and conditions. This will force you to get comfortable with your system and understand what it feels like to lose 5 times in a row for example. This way you will react better when you face these conditions in real trading.\\
\\
\underline{Triple Screen Trading System:}\\
- Applies three screens or filters to every trade: trend-following indicators on long-term charts, counter-trend oscillators on intermediate charts, and order entry technique.\\
- No one single indicator can tell you everything about the market and indicators are often contradictory, both across indicators and using the same indicator across different time frames.\\
- Pick your favorite time frame (mine is 1-hour) Triple Screen calls that the \textbf{intermediate} time frame (wave). The \textbf{long-term} time frame (tide) is one order of magnitude (by the factor of five) longer (1-day) and the \textbf{short-term} time frame (ripple) is one order of magnitude shorter (10-mins).\\
- Once you select your intermediate time frame, you may not look at it until you examine the longer-term time frame and make your strategic decisions there. This allows you to only trade in the direction of the tide (trend of the long-term chart).\\
- It uses the waves to go against the tide for entering positions.\\
- \textbf{First Screen:} Market Tide. Identify the long-term trend to eliminate either buy or sell side. Swim with the tide or stay out of the water.\\
- \textbf{Second Screen:} Market Wave goes against the tide. Applies oscillators to the intermediate charts to identify deviations from the long-term trend. When the long-term trend is up, only take signals from the oscillators that agree (i.e. tell you to buy).\\
- \textbf{Third Screen:} Entry Technique. Lots of leeway here. Most common is to go to your short-term chart and look for a ripple in the direction of the tide. The biggest consideration is the placing of your stops so that you don't get taken out by noise.\\
- Before you enter a trade, write down three numbers: the entry, the target, and the stop.\\
- Triple screen calls for setting profit targets on long-term charts and stops on the intermediate charts.\\
\\
\underline{The Impulse System:}\\ 
- A model of the market based on two ideas \textbf{inertia} and \textbf{power}.\\
- Inertia is measured by the slope of the fast EMA while power is measured by the slope of the MACD-Histogram.\\
- The Impulse System is a censorship system telling you what not to do (buy or sell).\\
- When both inertia and power agree it is acceptable to take actions, when they diverge it is time to sell or stay on the sidelines.\\
- Can work in both long and intermediate time frames for entering trades.\\
- Psychology note: The Impulse system encourages you to enter cautiously but exit fast. Beginners tend to do the opposite; jump into trades and then take forever to exit, hoping for the market to turn in their favor.\\
\\
\underline{Channel Trading System:}\\
- Channels help us anticipate where those support and resistance levels are likely to be encountered.\\
\textbf{Assumptions:} 
\begin{enumerate}
   \itemsep0em
   \item Value is defined as the zone between a short and a long moving average.
   \item Prices oscillate above and below value.
   \item If we have means to define value and measure an average oscillation we have a trading system.
\end{enumerate}
- Typically you build the channels around a longer EMA (26-day for example).\\
- There can be multiple levels of channels: 1-$\sigma$, 2-$\sigma$, 3-$\sigma$.\\
- A well drawn channel contains 90\% - 95\% of prices for the previous 100 bars.\\
- Channel trading is primarily a mean reversion strategy.\\
\section*{Chapter 8: Risk Management}
- Count ticks instead of dollars it will help with the emotions of trading.\\
- Two main factors of risk control: size of trades, where you place your stops.\\
- \textbf{The 2\% Rule prohibits you from risking more than 2\% of your account equity on any single trade.}\\
- Note: this is max loss not max size of a trade. This is the max size your allowed to have your stop at.\\
- Set a maximum monthly draw down for you account.\\
- \textbf{The 6\% Rule prohibits you from opening any new trades for the rest of the month when the sum of your losses for the current month and the risks in open trades reach 6\% of your account equity.}\\
- Higher levels of risk, impair our ability to perform.\\
- After a large draw down gradually work your way back by starting with a low amount of risk and working your way up in size.\\
- Traders do worse on their own because they have no manager. No one to hold them accountable when they violate systems or risk management.\\
\section*{Chapter 9: Practical Details}
- In order to find good trades, you need to \textbf{define the pattern} you want to trade. Your scanner must look for this pattern.\\
- For every trade you need a planned: entry level, profit target, and stop. Your potential reward should be at least twice as big as your risk.\\
- Moving averages and channels help set profit targets for swing trades.\\
- Placing Stops
\begin{itemize}
   \itemsep0em
   \item Place the stop outside the zone of market noise: standard deviation, ATR, IV.
   \item Don't place you stops at obvious levels: round numbers, previous resistance/support, previous highs/lows.
   \item Don't let a winning trade turn into a loss. Move your stops, but only in the direction of your trade.
\end{itemize}
\underline{Evaluating Trades:}\\
- It is instructive to give various aspects of your trades grades in order to improve the quality of the trades you make.\\
- Your buy grade is based on the location of your entry, relative to the high and low of the bar in which you bought.\\
\[Buy Grade = \frac{high - buypoint}{high - low}\]
- Sell grade is the equivalent inverse.\\
\[Sell Grade = \frac{sell point - low}{high - low}\]
- The trade grade can be given, not based on money made or lost, but on the entry and exit points.\\
\[Trade Grade = \frac{sell - buy}{channelHigh - channelLow}\]
- Be sure to add rules for liquidity and other filters when scanning for patterns.\\
\section*{Chapter 10: Record Keeping}
- Good record keeping is akin to good discipline.\\
- Three Components of Record Keeping:
\begin{enumerate}
   \itemsep0em
   \item Discipline begins with doing your homework. A set of tasks to prepare yourself for each trading day.
   \item Discipline is reinforced by writing down your trade plans.
   \item Discipline culminates in executing those plans and completing trade records.
\end{enumerate}
\underline{Homework:}\\
- It pays to have a morning routine for the market: a sequence of steps for touching base with the key factors that may dominate today's trading.\\
- Filling out a spreadsheet of key market components (indexes, vix, oil, rates, etc.) is one such routine.\\
- After the spreadsheet, review open trades and make stop adjustments as necessary.\\
- \underline{Elder's Homework}
\begin{itemize}
   \itemsep-0.5em
   \item Check Far East Markets
   \item Check Europe Markets
   \item Economic Calendar
   \item Marketwatch (news)
   \item Euro
   \item Yen
   \item Oil
   \item Gold (BTC)
   \item Bonds
   \item Baltic Dry Index (World Economy Indicator)
   \item NH-NL Index
   \item VIX
   \item S\&P 500
   \item Daily Value - of SPY based on above or below channel
   \item Force Index
   \item Expectation of SPY candle - Measures his own expectation of movement
   \item Summary of how he'll trade today
\end{itemize}
- \textbf{Are You Ready to Trade:} Take a 30-sec psychological test before each session.\\
\begin{center}
   \begin{tabular}{|c|c|c|c|c|c|}
      \hline
      Rating & Physically & Mood & Schedule & Homework & Yesterday \\
      \hline
      0 & Ill & Poor & Busy & Not Prepared & Lost Money \\
      \hline
      1 & Average & Average & Normal & Middling & Unch \\
      \hline
      2 & Excellent & Great & Open & Prepared & Made Money \\
      \hline
   \end{tabular}
\end{center}
- A plan for any trade must include what strategy you will use. You can have different rubrics for different strategies.\\
- A really good plan (rubric) will include the ability to measure the trade quality, before you trade.\\
- \textbf{If a seemingly attractive trade fits no trading system, then there is no trade.}\\
- A Trade Apgar requires clear answers to five questions that go to the heart of a trading strategy. Normally on a 0, 1, 2 point scale. 7 or higher is considered a good score.\\ 
- Each system will require a different "Apgar" test to see if the trade is good according to that system.\\
\\
\underline{Trade Journal:}\\
- Start as a diary and don't be afraid to let it evolve.\\
- You will have the Trade Plan and HW records so the Journal should provide additional context to why the trades were go/no-go.\\
- Reasoning behind stop levels and certain ratings, etc.\\
- Reviewing your trades a month out will help you learn and refine your rating skills.\\
- Should have some way to ultimately review equity curves, since that is the overall goal.\\
\end{document}
